\appendix
\section{Appendix}
\label{appendix}
%===============================================================
\subsection{AI Use Declaration}
In accordance with UNIL Lausanne's guidelines for responsible AI usage in academic writing, the table below details the artificial intelligence (AI) tools employed in the creation of this report to promote transparency and academic integrity.

\begin{table}[h!]
    \centering
    \caption{AI Use Declaration}
    \label{tab:ai_use_declaration}
    \begin{tabular}{@{}p{3cm} p{3cm} p{3cm} p{7cm}@{}}
        \toprule
        \textbf{AI-Based Tool} & \textbf{Use Case} & \textbf{Scope} & \textbf{Remarks} \\ \midrule
        Gemini 2.5 Pro \newline Last access: 07.08.2025 & Text generation & Selected sections & Used for brainstorming and clarifications. \\
        Claude Sonnet Thinking 3.7 \newline Last access: 07.08.2025 & Text generation & Selected sections & Used for grammatical errors and style improvements. \\
        GPT 4.1 \newline Last access: 07.08.2025 & Text generation & Selected sections & Used for code suggestions and editing. \\
        \bottomrule
    \end{tabular}
\end{table}


\subsection{GitHub Repository}
All code used for data analysis, as well as all images and reports referenced in this document, are available in the following GitHub repository:

\begin{itemize}
    \item \textbf{Repository:} \url{https://github.com/your-username/your-repo-name}
    \item \textbf{Contents:}
    \begin{itemize}
        \item \texttt{analysis\_code.ipynb}: Jupyter notebook containing all analysis code
        \item \texttt{data/}: Folder containing all datasets used
        \item \texttt{reports/}: Reports of the different subgroup analyses
        \item \texttt{figures/}: All generated figures
        \item \texttt{requirements.txt}: List of required Python packages
    \end{itemize}
\end{itemize}

For further details or to reproduce the results, please refer to the repository above.
